% JuliaCon proceedings template
\documentclass{juliacon}
\setcounter{page}{1}

\begin{document}

\title{Bringing Julia to Health: A Concise Overview of JuliaHealth} 

\author[1]{Jacob S. Zelko}
\author[2]{Carlos Castillo-Passi}
\author[3]{Jakub Mitura}
\author[4]{Divyansh Goyal}
\author[5]{Kosuri Lakshmi Indu}
\author[6]{Jay Sanjay Landge}

\affil[1]{CEPHIL, University of Saskatchewan}
\affil[2]{Department of Radiology, Stanford University}
\affil[3]{University Clinic for Radiology and Nuclear Medicine, Otto-von-Guericke-Universität}%, Magdeburg, Germany}
\affil[4]{Guru Gobind Singh Indraprastha University}
\affil[5]{BVRIT Hyderabad College Of Engineering for Women}
\affil[6]{Indian Institute of Technology (IIT), Hyderabad}

\maketitle

\keywords{health informatics, medical imaging, open source communities}

\begin{abstract}
JuliaHealth is a open science volunteer ecosystem within the Julia community that seeks to bring open source high performance computing methods and software from Julia to health research.
In this paper, we provide a concise overview of its medical imaging and observational health software ecosystems and its presence within the broader Julia community.
We conclude this paper with future JuliaHealth development goals and directions for growth.
\end{abstract}

\section{Introduction}

Founded in the early $2020$'s, JuliaHealth was created to support bringing high performance computing methods from within Julia to various aspects of health research.
Originally, JuliaHealth was started to centralize and discuss efforts around the COVID19 pandemic within the broader Julia community. 
Since the end of the pandemic, JuliaHealth has continued to grow with over $300$ members worldwide and have an impact in various professional venues. \cite{zelko2025juliahealth, Zelko2023Julia}
As a result of this growth, several communities have organically emerged within JuliaHealth to meet the multifaceted needs of health research.
The purpose of this paper is to highlight some of these communities and their current capabilities, overarching near-term and long-term community goals, and how JuliaHealth integrates with the broader Julia community.

\section{Ecosystems within JuliaHealth}

Within JuliaHealth, there are several growing ecosystems.
Here, we particularly highlight the medical imaging and observational health analysis ecosystems which are the more mature areas within JuliaHealth.
We conclude by broadly gesturing to other developing aspects of JuliaHealth or ecosystems.
All packages mentioned are considered under development with each under development at different rates.

\subsection{Medical Imaging}

The JuliaHealth medical imaging ecosystem provides high-performance, open-source tools for end-to-end medical imaging research. 
The ecosystem spans physics simulations, data acquisition, and reconstruction to image processing, segmentation, and interpretation.

\subsubsection{Image Acquisition and Simulation}  

The core package for medical imaging simulation in JuliaHealth is KomaMRI.jl \cite{castillo-passi_komamri_2023}, an open source framework for general MRI simulation on CPUs and GPUs. It solves the Bloch equations with high numerical precision and provides a flexible playground for designing and optimizing MRI pulse sequences with or without physiological motion. It has already been used for cardiac sequence optimization at low magnetic field strengths \cite{castillo-passi_highly_2025}, synthetic dictionary generation for quantitative techniques, evaluations of hardware imperfections, and experimental imaging techniques for measuring neuron currents. Its high performance simulation core enables fast and reproducible exploration of sequence behavior, motion effects, and hardware limitations before scanner implementation, making it a powerful tool for physics based optimization and machine learning applications.

\subsubsection{Image Processing, Segmentation and Interpretation}

For post-acquisition methods, JuliaHealth includes a comprehensive suite of packages for image I/O, visualization, segmentation evaluation and workflow orchestration. 
For example, MedPipe3D.jl orchestrates MedImages.jl, MedEye3D.jl, and MedEval3D.jl into a comprehensive workflow for data visualization, inspection and annotation, and segmentation and segmentation-evaluation of 3D/4D images and other volumetric data (e.g. MRI, CT, PET).
Beyond orchestration, I/O packages such as DICOM.jl can offer support for other imaging files such as DICOM images, metadata parsing, anonymization and conversion to numerical arrays.
Together JuliaHealth offers a complete post-acquisition research environment that supports reproducible and scalable medical-imaging analysis and interpretation.

\subsection{Observational Health}

The JuliaHealth medical imaging ecosystem provides high-performance tools for end-to-end medical imaging research. 
The ecosystem spans physics simulations, data acquisition, and reconstruction to image processing, segmentation, and interpretation.

The JuliaHealth observational health ecosystem offers capabilities that support patient population analysis, phenotype definition management, interoperability with OHDSI tools, OMOP CDM database handling, and study creation and maintenance.
Packages like OMOPCDMCohortCreator.jl, OMOPCommonDataModel.jl, and OHDSIAPI.jl assist researchers in defining patient populations, managing OMOP CDM data, and integrating with external OHDSI tools.
Furthermore, building on OMOP CDM databases, OMOPCDMPathways.jl exists to map and analyze a patient's care continuum while OMOPCDMFeasibility.jl characterizes patient cohorts based on phenotype definitions and statistical profiels of populations across covariates of interest.
Finally, HealthBase.jl provides templates for studies such as observational health studies and a common interface for packages to dispatch upon to bring unity to workflows across JuliaHealth.
With these tools, JuliaHealth offers several options for conducting transparent, large-scale observational health analyses across institutions.

\subsection{Other Developing Ecosystems}

Beyond these two ecosystems, many parts of JuliaHealth are actively under development.
For example, as it relates to simulation of biological and physiological processes, KomaMRI.jl supports MRI physics, Thunderbolt.jl simulates cardiac electrophysiology and mechanics, and BloodFlowTrixi.jl can be used to explore hemodynamics. 
For interacting with third-party bioinformatics platforms, several API wrappers and specialized tools exist to bring relevant data or information from biomedical literature and chemical databases into one's workflow.
These tools support programmatic retrieval and analysis of such information via tools like PubMedMiner.jl for accessing literature from PubMed/MEDLINE or PubChemMiner.jl for crawling chemical structures and properties from PubChem for crawling chemical structures and properties.
As JuliaHealth continues to grow, such efforts in simulation modeling and bridging to various bioinformatics resources will serve to expand the JuliaHealth into new ecosystems and further stages of development.

\section{Discussion}

While JuliaHealth is still young, it has a steady presence within the broader Julia community. 
Here, we highlight strategic partnerships JuliaHealth has been leveraging across the Julia ecosystem and community involvement. 

\subsection{Community Partnerships}

Through cultivation of Julia community relationships, JuliaHealth has been able to expand its capabilities and support trans-organizational efforts in developing novel applications across disciplines. 
As a result, JuliaHealth has worked consistently with community organizations such as: 

\begin{itemize}
    \item \href{https://github.com/BioJulia}{\textbf{BioJulia}} - Fast, open, extensible software for bioinformatics and computational biology.
    \item \href{https://github.com/JuliaEpi}{\textbf{JuliaEpi}} - Open research community to develop an epidemiological modeling ecosystem written in Julia.
    \item \href{https://github.com/JuliaImageRecon}{\textbf{JuliaImageRecon}} - General purpose image reconstruction and computational imaging packages.
    \item \href{https://github.com/MagneticParticleImaging}{\textbf{MagneticParticleImaging}} - Julia ecosystem for Magnetic Particle Imaging (MPI) simulation framework.
    \item \href{https://github.com/MagneticResonanceImaging}{\textbf{MagneticResonanceImaging}} - Collection of packages for MRI research and application (simulation, reconstruction and analysis).
    \item \href{https://github.com/JuliaSurv}{\textbf{JuliaSurv}} - Survival analysis tools and packages in Julia.
\end{itemize}


\subsection{Community Outreach}

JuliaHealth also seeks to foster a safe and welcoming environment through multiple different communication platforms and community mentoring.
We accomplish these goals by providing and promoting: 1) monthly community meetings 2) community programs such as Google Summer of Code (GSoC) or NumFOCUS-sponsored events 3) discussion forums such as the Julia Discourse, Julia Slack, and Julia Zulip 4) package development and planning on code hosting utilities such as the JuliaHealth GitHub 5) the JuliaHealth Blog to highlight ongoing work and community projects.  
 
\section{Future Community Development}

As JuliaHealth continues to develop new tools and methodologies, some future areas of development are: 1) combining census microdata alongside observational health data for exploring social determinants of health 2) Interactive community dashboards with custom JuliaHealth modules to host and present analysis results. 3) using JuliaHealth generative AI and large language models to support health research analysis and tool development.

\section{Conclusion}

In conclusion, JuliaHealth provides an infrastructure for various aspects of health research. 
As the ecosystem continues to grow, it will continue to provide a centering presence within the Julia ecosystem to enable comprehensive research workflows and future software development. 
As active development continues, JuliaHealth offers a practical and extensible foundation for bringing high performance and high quality to health research.

\section{Acknowledgements}

We warmly thank Dr. Dilum Aluthge for initiating JuliaHealth, the JuliaHealth and broader Julia community, and early contributions by Malina Hy, Varshini Chinta, and Fareeda Abdelazeez.

\bibliographystyle{juliacon}
\bibliography{ref.bib}

\end{document}

% Inspired by the International Journal of Computer Applications template
